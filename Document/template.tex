\documentclass[a4paper,openright,12pt]{article}
\usepackage[spanish]{babel} % espanol
\usepackage[utf8]{inputenc}
\usepackage{graphicx} % graficos
\usepackage{fancyhdr}
\usepackage[margin=2.5cm, top=2.5cm,includefoot]{geometry}
\usepackage[hidelinks]{hyperref}
\usepackage[table,xcdraw]{xcolor}
\usepackage{minted}
\usepackage[
backend=biber,
style=alphabetic,
sorting=ynt
]{biblatex}
\addbibresource{sample.bib}


% VARIABLES 
\newcommand{\logoITB}{./data/img/ITB_Logo_417.png}
\newcommand{\autor}{Marc Hortelano }
\newcommand{\startDate}{Maig 2021}

\newcommand{\LogoGPTS}{./data/img/Logo_ptls.png}

\newcommand{\cronologia}{./data/img/Cronograma vertical.png}
\newcommand{\scriptEstruc}{./data/img/Estructura_Script.png}

% ADICIONALES
%\renewcommand{\contentsname}{Index}
\setlength{\headheight}{50.2pt}
\pagestyle{fancy}
\fancyhf{}
\lhead{\includegraphics[width=4cm]{\logoITB}}
\chead{ASIX c, Maig del 2021}
\rhead{\includegraphics[width=2cm]{\LogoGPTS}}
\rfoot{\thepage}
\lfoot{\autor}
\renewcommand{\headrulewidth}{1pt}
\renewcommand{\headrule}{\hbox to\headwidth{\color{blue}\leaders\hrule height \headrulewidth\hfill}}
\setlength\parindent{0pt}


% DOCUMENT START
\begin{document}

\begin{titlepage}

    \begin{center}
        \begin{figure}[htb]
            \begin{center}
                \includegraphics[width=4.5cm]{\LogoGPTS}
                \includegraphics[width=6cm]{\logoITB}
            \end{center}
                \vspace*{0.1in}
        \end{figure}

        \vspace*{0.2in}
        \begin{Large}
            \textbf{GPenT0ols} \\
        \end{Large}
        \vspace*{0.3in}
        \begin{large}
            Aquest projecte esta creat per en \autor en el curs de Grau Superior\\
            \vspace*{0.3in}
            \startDate \\
        \end{large}
        \vspace*{0.3in}
        \rule{80mm}{0.1mm}\\
        \vspace*{0.1in}
        \begin{large}
        Tutor: Victor Marquina\\
        - \\
        \end{large}
            Grau Superior - Administració de Sistemes Informàtics en Xarxa\\
        \vspace*{0.15in}
            especialitat en Ciberseguretat \\
        \vspace*{0.6in}
    \end{center}

    \end{titlepage}
    % ---- %
    \clearpage
    \tableofcontents
    \clearpage
    % ----- %
    \setcounter{page}{2}
    \section{Abstract}
When you start a project you look for precision, but I think the most important thing for me is how you do it.
It's how you do it. My main goal is to enjoy learning about this subject as much as possible.
The topic I have chosen has been very complicated to define because it can be very broad and sometimes you can get saturated with so much information you find on the Internet.
First of all I am going to write about my motivations. Why I have chosen this final project and not another one, since I have had in mind several alternatives and all of them very similar.

Then I would like to show how I have done the project, the steps I have followed and show the final result. Add that my goal of this project is that other people can use it as they want. Which I will show how I have planned it so that it has no security problems as well as how to use it.

As I said before I hope to enjoy and assimilate as much as I can doing this kind of work because I think it helps me a lot to get out of my comfort zone and this kind of challenges makes learning much more fun.

\pagebreak
    \section{Introducció}
   
    Feia temps que volia crear-me una eina automatitzada per poder realitzar pentestings, a més a més treballar una mica el desenvolupament web i conèixer noves tecnologies també estava dins el meu pla.
La idea del projecte es realitzar un entorn de pentesting de forma gràfica mitjançant una web. Aquest entorn és una infraestructura en Docker on recolliré totes les eines que son útils per a mi. Em permetrà realitzar una auditoria de pentesting i desplegar-la de forma còmode i ràpida.\par
Un pentest consisteix en realitar un procés d'avaluació coordinat. La prova implica una varietat d'elements, però per a simplificar l'explicació, un individu o equip contractat accedeix al sistema, avaluaria tot el sistema cercant vulnerabilitats o febleses a través d'una metodologia predefinida, aquestes vulnerabilitats són explotades de manera controlada i permet identificar el risc per a l'organització. \par\par 
La idea s'origina en la necessitat d'automatitzar i les ganes d'aprendre o millorar un llenguatge de programació, concretament Python. Com he comentat anteriorment aquest projecte no ha set la primera idea que he tingut, ja que abans han passat varies molt similars i molt interessants. Una d'aquestes i que m'ha ajudat molt a obtenir aquesta idea final, era crear un plugin del Framework caldera.
Caldera és un framework que utilitza les tàctiques de MITRE ATT\&CK, i s'utilitza tant per red team com per blue. Aquest inici em va permetre entendre una mica el funcionament de com estava programat.
Però aquesta idea no m'acabava de convecer, ja que principalment volia crear una eina per realitzar pentesting i no de red team. Molta gent ho engloba tot i realment són metodologies diferents. Més endavant explicaré que signifiquen aquests conceptes, però detallar aquesta part ho he trobat important perquè ha set de gran ajuda per obtenir la idea final.
Finalment vaig decidir crear la meva pròpia eina des de zero, anomenada GPenTools. Com he dit, ho escriure principalment amb Python tant l'entorn gràfic com la majoria de scripts. El que faré serà una web on hi hagui una shell interactiva i tindré totes les eines de pentesting que consideri. Aquesta web et permetrà modificar ràpidament el script i també crear tasques automatitzades.
Per la creació d'aquesta eina m'ha servit d'inspiració altres eines com \cite{1} \href{https://github.com/byt3bl33d3r/CrackMapExec}{\color{blue}crackmapexec} i \cite{2}\href{https://github.com/Tib3rius/AutoRecon}{\color{blue}autorecon}.\par
Tota la informació la compartiré en el meu repositori de Github, ja que ho vull fer de forma pública i que qualsevol persona la pugui utilitzar. \\

    \begin{center}
        Enllaç del repositori: \href{https://github.com/Th3FirstAvenger/GPenT0ols}{\color{blue}https://github.com/Th3FirstAvenger/GPenT0ols}\\
    \end{center}
\pagebreak
    \section{Objectius}
    Per poder executar aquest projecte, ha set necessari complir uns requisits mínims. Aquest requisit seria comptar amb les bases de Linux i tenir uns coneixements de pentesting. Els objectius que m'agradaria assolir serien aconseguir una primera experiència desenvolupant una eina automatitzada totalment pròpia amb el llenguatge de programació de python, poder aprofitar aquesta eina per implementar-la en les auditories i seguir treballant amb ella per aconseguir millores.
Per complir els objectius ha set necessari una primera fase de planificació, aquesta part consider-ho que ha set la més important ja que soc una persona que li agrada realitzar inversions a llarg termini i tenir una bona preparació.

Com bé deia Abraham Lincoln " Give me six hours to chop down a tree and I will spend the first four sharpening the ax. " Dona'm sis hores per tallar un arbre i em passaré les quatre primeres afilant la destral.

Al tenir poca experiència en desenvolupar un projecte de programació a llarg termini, com es normal he comès alguns errors i imprevistos. Sent veritat, que gràcies a l'haver planificat una idea inicial i ben estructurada he rectificat a temps sense que hagi complicat molt el projecte. Quan dic que no tinc experiència vull fer referència que jo en l'àmbit de programació, només he realitzat scripts automatitzats i creació d'algun exploit web. Aquests a nivell bàsic.
\pagebreak
    \section{Anàlisi de la programació}
    La planificació del projecte ha estat estructurada de la següent forma: 
    \begin{figure}[h]
        \centering
          \includegraphics[scale=0.4]{\cronologia}
            Figura 1: Cronograma
    \end{figure}
    \pagebreak
    \section{Resultats}
    En aquest apartat compartiré el codi i les instruccions per poder utilitzar la eina GPenT0ols. Tota la informació es troba en el meu repositori de github que he anunciat anteriorment. 

    Per realitzar la instal·lació recomano que es faci amb docker ja que et permet tenir la ultima versió del projecte amb totes les eines i llibreries necessaries. Apart aquesta eina unicament ocupa 3 GB en la que cada vegada que es utilitzada s'elimina automaticament a no se que utilitzis configuracions que no s'indiquen a la instal·lació. 

\subsection{Instal·lació}
Primer de tot es necessari tenir-ho descarregat

\begin{minted}
[
frame=lines,
framesep=2mm,
bgcolor=white,
fontsize=\footnotesize,
]
{bash}
git clone https://github.com/Th3FirstAvenger/GPenT0ols.git /opt/GPenT0ols
cd /opt/GPenT0ols
\end{minted}

\subsubsection{Per utilitzar python3}
En aquest cas ja tindriem l'eina descarregada. Faltaria descarregar i instal·lar les dependències minimes per poder utilitzar l'eina. Per tenir-ho es pot fer us de la següent comanda: \par 
\textit{* Aquesta comanda es valida per kali linux, en altres distribucions es probable que hagis de realitzar una instal·lació manual. }


\begin{minted}
[
frame=lines,
framesep=2mm,
bgcolor=white,
fontsize=\footnotesize,
]
{bash}
apt-get update && \
        apt-get install -y \
    python3 \
    python3-pip \
    git \
    nmap \
    masscan \
    smbmap \
    whatweb \
    snmp \
    wget \
    nbtscan \
    wpscan \
    enum4linux \ 
    nikto \
    ffuf \
    golang \
    python3-venv \
    crackmapexec \
    seclists 
\end{minted}

El següent pas ja es realitzar l'execució. 


\begin{minted}
[
frame=lines,
framesep=2mm,
bgcolor=white,
fontsize=\footnotesize,
]
{bash}
python3 AutoGPenT0ols.py -h 
\end{minted}

    \subsubsection{Docker (Recomenada)}
    A continuació mostraré els passos per instal·lar l'eina amb docker.
Per la creació de la imatge haurem d'executar la següent comanda.
\begin{minted}
[
frame=lines,
framesep=2mm,
bgcolor=white,
fontsize=\footnotesize,
]
{bash}
docker build -t capitanj4ck/gpent0ols .
\end{minted}
Per fer l'execució es bestant facil i com he indicat abans d'aquesta forma l'eina no es guarda i s'elimina cuan acaba l'execució. Per iniciar utilitzariem la següent comanda:
\begin{minted}
[
frame=lines,
framesep=2mm,
bgcolor=white,
fontsize=\footnotesize,
]
{bash}
docker run --rm -it -v /tmp/gpt_report:/tmp/gpt_report capitanj4ck/gpent0ols -h
\end{minted}

Recomano afegir el següent alies per fer l'execució d'una forma més ràpida: 

\begin{minted}
[
frame=lines,
framesep=2mm,
bgcolor=white,
fontsize=\footnotesize,
]
{bash}
alias gpt="docker run --rm -it -v /tmp/gpt_report:/tmp/gpt_report capitanj4ck/gpent0ols"
\end{minted}

\pagebreak
\subsection{Funcionament}
Primer de tot explicaré com esta estrucurada l'eina. M'agradaria donar molta més énfasi a l'apartat del codi ja que la web encara enstà en desenvolupament. He prioritzat aprendre el funcionament de python abans que flask, l'objectiu inicial era tenir una eina propia.\par 
A l'executar l'eina ens permet realitzar diferents funcions. Com he dit anteriorment, el que ens permet realitzar aquesta eina es enumerar qualsevol equip de forma bàsica. 
Com que la meva intenció es seguir treballant amb aquesta eina, inicialment vaig planificar una estructura que em permetés poder desenvolupar el projecte a llarga durada i no fos únicament un fitxer python que ho realitzes tot. Un dels objectius era entendre com desenvolupar una eina de forma professional i no únicament teclejar i entendre el codi. Per aquest motiu vaig estar preguntat a professionals i veien diferents programes realitzats en python. D'aquesta forma em van servir d'inspiració diferents programes i amb aquesta ajuda vaig aconseguir planificar una estructura que fins el dia d'avui no he tingut molt problemes.

Per tant procediré en mostrar com la tinc estructurada.    
\begin{figure}[h]
          \includegraphics[scale=0.5]{\scriptEstruc}
        \centering\par
            Figura 2: Estructura de l'eina
    \end{figure}

    \subsubsection{Codi}
Veiem que inicialment tenim un fitxer main, aquest em permet cridar les diferents llibreries creades i cada llibreria té les llibreries necessàries i així no omplo el primer fitxer de codi que únicament em pot servir per a un servei.
El contingut d'aquest fitxer seria el següent:
\subsubsubsection{AutoGPenT0ols.py}
\begin{minted}
[
frame=lines,
framesep=2mm,
baselinestretch=1.2,
bgcolor=white,
fontsize=\footnotesize,
linenos
]
{python}
#!/usr/bin/python3 
## 
                           #                                                                                                                                                                                         
                           ##                                ###                                                                                                                                                     
  ###### #######  ######## ###  ## ######## ####### #######  ###         ####                                                                                                                                        
 ###           ##          #### ##    ###   ##   ##       ## ###        ###                                                                                                                                          
 ###  ##  ######   ####### #######    ###   ##   ##  ##   ## ###        ###                                                                                                                                          
 ###  ##  ###      ###     ### ###    ###   ##   ##  ##   ## ###        ###                                                                                                                                          
  ######  ###      ####### ###  ##    ###   #######   #####  ####### #####                                                                                                                                           
                                 #                                                                                                                                                                                   
# Author : CapitanJ4ck
##

import signal
from sys import exit
import os
import time
import subprocess
from pwn import *
from gptools_cli import gen_cli_args
from services.recon import recon
from services.web import web
from services.smb import smb
from services.ldap import ldap
from services.ftp import ftp

## Detect Contrl + C 
def signal_handler(key, frame):
    # Handle any cleanup here
    exit = log.progress("SIGINT or CTRL-C detected.")
    exit.status("Exiting...")
    time.sleep(1)
    exit.failure("Exiting gracefully")
    sys.exit(1)

signal = signal.signal(signal.SIGINT, signal_handler)


# Vars 

# Make directories

def mdir(dir_name):
    
    directory_created = True

    try:
        os.makedirs(dir_name)
    
    except OSError:
        directory_created = False
    
    return directory_created

# Managment function, we can build directories for save outputs

def build_infraestucture(dir_name,output):

    infra = log.progress("Managment")
    
    infra.status("Building structure on ")

    
    if not os.path.exists(dir_name):
        directory_created = mdir(dir_name)
        if directory_created: 
            infra.success("Succesfully created the directory {}".format(dir_name))
        else: 
            infra.failure("Creation of the directory {} failed".format(dir_name))
    else: 
        infra.success("Directory {} already exist".format(dir_name))

def run(dir_file, command,service,debug):
    out = dir_file + "out.txt"
    err = dir_file + "err.txt"
#    parsed_command = command ## Util working other commands 
    parsed_command = []
    
    for c in command:
        parsed_command.append(c.replace('·',' '))
    

    p = log.progress(service)
    
    with open(out,'w+') as fout:
        with open(err,'w+') as ferr:
            
            p.status("Running")           
            try: 
                out=subprocess.call(parsed_command,stdout=fout,stderr=ferr)
                # reset file to read from it
                fout.seek(0)
                # save output (if any) in variable
                output=fout.read()
                if debug:
                    print(output) 

                # reset file to read from it
                ferr.seek(0) 
                # save errors (if any) in variable
                errors = ferr.read()
                if out != 0: 
                    p.failure("Something wrong")
                else: 
                    p.success("Succesfully!")
            except: 
                p.failure("Command timeout")
            
def main():
    
    # Get args from Namespace type 
    args = vars(gen_cli_args())
    
    # Get service
    service = args['services']
    target = args['target'] 
    debug = args['verbose'] 
    out_path = os.path.join(args['path'],(os.path.join(args['target'],service)))
    config_path = os.path.join(os.getcwd(), os.path.join("data",service))
    
    web_path = out_path # CHANGE WHEN WEB WORKS

    # Build infraestucture for save output
    build_infraestucture(out_path,config_path)

    #print(args) # debug
    # Start progress
    service_progress = log.progress(service)
    
    ## Recon scanner
    if 'recon' == service: 
        scanner = recon(args,config_path,out_path)
    ## Web scanner
    elif 'web' == service: 
        scanner = web(args,config_path,out_path)
    ## smb scanner
    elif 'smb' == service: 
        scanner = smb(args,config_path,out_path) 
    ## ldap scanner
    elif 'ldap' == service: 
        scanner = ldap(args,config_path,out_path) 
    ## ftp scanner
    elif 'ftp' == service: 
        scanner = ftp(args,config_path,out_path) 
    
    print("-- 𝒞𝒽𝑒𝒸𝓀 𝒢𝒫𝑒𝓃𝒯𝒪𝑜𝓁𝓈 --")
    for description, command in scanner.items():
        service_progress.status("{}".format(command))
        if args['show_commands']:
            print(command)
            time.sleep(2) # Check without exec 
        else: 
            run(web_path,command.split(),description,debug)

if __name__ == '__main__':
    main()
\end{minted}

En aquest fitxer mostrat anteriorment el que fa es agafar els arguments passats pel fitxer gpentools\_cli.py. A partir del que passa l'usuari executarà una cosa o altra. Més endavant mostraré el que mostra i el seu funcionament.\par

\subsubsubsection{gptools\_cli.py}

Aquest fitxer fa ús del mòdul argparse, aquest mòdul em permet agafar els arguments de forma fàcil i compatible de moltes formes. També he escollit aquest mòdul perquè em permet crear com submenús i aprofundir més en detall cada servei. Aquest mòdul el comparteixo a la webgrafia, ja que ha set complexa entendre el seu funcionament però finalment he aconseguit implementar-ho.
\par

El contingut del fitxer gpentools\_cli és el següent: 

\begin{minted}
[
frame=lines,
framesep=2mm,
baselinestretch=1.2,
bgcolor=white,
fontsize=\footnotesize,
linenos
]
{python}
import argparse
import sys
from argparse import RawTextHelpFormatter
from lib.core.__version__ import __version__
from helpers.logger import highlight

def gen_cli_args():

    VERSION  = __version__
    CODENAME = 'CapitanJ4ck'


    parser = argparse.ArgumentParser(description="""
                           #                                                 
                           ##                                ###             
  ###### #######  ######## ###  ## ######## ####### #######  ###         ####
 ###           ##          #### ##    ###   ##   ##       ## ###        ###  
 ###  ##  ######   ####### #######    ###   ##   ##  ##   ## ###        ###  
 ###  ##  ###      ###     ### ###    ###   ##   ##  ##   ## ###        ###  
  ######  ###      ####### ###  ##    ###   #######   #####  ####### #####   
                                 #                                           
                                 {}: {}
                                 {}: {}
""".format(highlight('Version', 'red'),
           highlight(VERSION),
           highlight('Codename', 'red'),
           highlight(CODENAME)),

                                    formatter_class=RawTextHelpFormatter,
                                    epilog="We are in... Let the hacking begin!")

    parser.add_argument("-t", type=int, dest="threads", default=100, 
                        help="set how many concurrent threads to use (default: 100)")
    parser.add_argument("--verbose", action='store_true', help="enable verbose output")
    parser.add_argument("--show-commands", action='store_true', help="Just show commands")
    parser.add_argument("--path",dest="path", default='/tmp/gpt_report/', 
                        help="Destination path (default: /tmp/gpt_report)")

    std_parser = argparse.ArgumentParser(add_help=False)
    std_parser.add_argument("target", nargs='?', type=str, 
                            help="(Target Required *) The target IP(s), range(s), 
                            CIDR(s), hostname(s), FQDN(s), file(s) containing a list of targets ")
   # Type scanner group
    scan_parser = argparse.ArgumentParser(add_help=False)
    scan_parser.add_argument(
            '--scanner',
            help='Select scanner (default : full_scanner)',
            nargs='?',
            default = 'full_scanner'
            )

   # Introduce what info you have
    have_info_parser = argparse.ArgumentParser(add_help=False)
    have_info_parser.add_argument(
            '--tags',
            help='What do you have? [Creds, NoCreds, Hash, Shell] (Default: NoCreds)',
            default = 'NoCreds',
            nargs='?'
            )

   # wordlist group
    wlist_parser = argparse.ArgumentParser(add_help=False)
    wlist_parser.add_argument("-w", metavar="WORDLIST", dest='wordlist', 
                            nargs='+', help="set wordlist  (Default SecList wordlist)")
   
   # ssl group
    ssl_parser = argparse.ArgumentParser(add_help=False)
    ssl_parser.add_argument(
            '--ssl',
            help='usage of SSL/TLS requests',
            action='store_true'
            )

   # credentials group
    cred_parser = argparse.ArgumentParser(add_help=False)
    cred_parser.add_argument("-u", metavar="USERNAME", dest='username', 
                        nargs='?', default=[], help="username(s) or file(s) containing usernames")
    cred_parser.add_argument("-p", metavar="PASSWORD", dest='password', 
                        nargs='?', default=[], help="password(s) or file(s) containing passwords")
    cred_parser.add_argument("-H", metavar="HASH", dest='HASH', 
                            nargs='?', default=[], help="Pass The hash")

    subparsers = parser.add_subparsers(title='services', dest='services', 
                                            description='available options')


    # Arguments Recon

    recon = subparsers.add_parser('recon', help='Initial recon', 
            parents=[std_parser,scan_parser])   
        ## Get new arguments and can introduce std_parser arguments
    recon.add_argument(
            '--all-ports',
            help='scan all ports',
            action='store_true'
            )
    
    recon.add_argument(
            '--ports',
            help='scan specific ports',
            nargs='?'
            )
    
    recon.add_argument(
            '--full',
            help='Full recon scan',
            action='store_true'
            )


    # Arguments WEB

    web = subparsers.add_parser('web', help='Web server scanner', 
                parents = [cred_parser, std_parser,wlist_parser,scan_parser,ssl_parser])
    
    web.add_argument(
            '--port',
            help='scan specific port (Default 80)',
            nargs='?',
            default = '80'
            )
    
    web.add_argument(
            '--file-path',
            help='Specify to find the requested resource and start the enumeration with that route (Default / )',
            nargs='?',
            default = '/'
            )
    

    web.add_argument(
            '--cms',
            help='What do you have? [Wordpress, Joombla, Drupal] (Default: NoCreds)',
            default = 'NoCreds',
            nargs='?'
            )
    # Arguments SMTP

    smtp = subparsers.add_parser('smtp', help='smtp enumeration')

    # Arguments SMB

    smb = subparsers.add_parser('smb', help='Enum smb', parents = [cred_parser,std_parser,have_info_parser])

    smb.add_argument(
            '--port',
            help='scan specific port (Default 445)',
            nargs='?',
            default = '445'
            )
    
    
    # Arguments FTP
    
    ftp = subparsers.add_parser('ftp', help='FTP enum', 
                    parents = [cred_parser,std_parser,have_info_parser,ssl_parser])

    ftp.add_argument(
            '--port',
            help='scan specific port (Default 21)',
            nargs='?',
            default = '21'
            )
    
    
    # Argumets LDAP

    ldap = subparsers.add_parser('ldap', help='LDAP enum', 
                        parents = [cred_parser,std_parser,have_info_parser])
    
    ldap.add_argument(
            '--port',
            help='scan specific port (Default 389)',
            nargs='?',
            default = '389'
            )

    # Arguments SNMP
    
    snmp = subparsers.add_parser('snmp', help='Enum SNMP', parents = [cred_parser,std_parser])

    if len(sys.argv) == 1:
        parser.print_help()
        sys.exit(1)

    args = parser.parse_args()
    
    check = vars(args)


    if not check['target']:
        if check['services'] == 'recon':
            recon.print_help()
        elif check['services'] == 'web':
            web.print_help()
        elif check['services'] == 'smb':
            smb.print_help()
        elif check['services'] == 'smtp':
            smtp.print_help()
        elif check['services'] == 'ftp':
            ftp.print_help()
        elif check['services'] == 'ldap':
            ldap.print_help()
        elif check['services'] == 'snmp':
            snmp.print_help()
        else: 
            parser.print_help()
        sys.exit(1)

    return args
\end{minted}

Una vegada s'han passat els arguments de forma correcta ja passen a executar els serveis. Cada servei té el seu fitxer i està creat com una llibreria. En el fitxer Main, veiem que crida unes funcions i aquestes funcions són les que es troben dins de cada servei. Sí que la metodologia és molt similar a cada fitxer però a poc a poc aniré implementant millores. Per exemple els fitxers recon i web són molt similars en canvi amb samba i ldap ja són diferents. El motiu es que he estat realitzant proves per veure quina seria la més eficaç i de moment crec que l'estructura que mantindré es la forma que estan configurats la segona opció.
En primer lloc la primera opció, fa ús dels fitxers de configuració .yaml però l'estructura es diferent. Podem observar que la primera opció s'ha de passar un argument i serà el mateix que cridarà en el fitxer yaml, en canvi la segona opció utilitzem sempre els mateixos tags i busca a través d'aquest tags. Si que pel servei de recon la millor opció es l'actual pero per l'apartat web seria modificar-ho i utilitzar els tags, però aquesta part la fa molt més complexa, ja que s'hauria de planificar quins serien aquest tags i com ho organitzaria per aquest motiu de moment ho mantinc com està.
\par

Seguidament mostraré el codi dels serveis. Aquests serveis són importats en l'inici del codi Main i permet fe ús de les funcions que es troben dins d'aquestes llibreries.
Les funcions són passar-l'hi la comanda que haurà d'executar, retorna un array amb la descripció de la comanda i la comanda.
La informació l'agafa del seu fitxer de configuració, el motiu que ho he realitzat d'aquesta forma es perquè em permet ampliar les comandes sempre que ho vegi convenient i no tindré cap problema, ja que al final retorno un array amb tota la informació.

Inicio la part dels serveis mostrant el codi i el seu fitxer de configuració. 

\subsubsubsection{recon.py}
\begin{minted}
[
frame=lines,
framesep=2mm,
baselinestretch=1.2,
bgcolor=white,
fontsize=\footnotesize,
linenos
]
{python}
#!/usr/bin/env python3
### This script enum all ports, services and vulnerabilities with nmap
import yaml
import os 



def recon(all_info, recon_path, out_path):
    target = all_info['target']
    info_data = all_info['scanner']
    
    # get path information
    recon_data = os.path.join(recon_path, "recon_config.yaml")
    command_info = {} 
    with open(recon_data, 'r') as unparsed:
        try:
            recon_data = yaml.safe_load(unparsed)
        except yaml.YAMLError as exc:
            print(exc)
        
        list_data = recon_data[info_data].keys()

        for options in list_data: 
            descr = recon_data[info_data][options]['description']
            out_file = os.path.join(out_path, '.'.join((options, 'txt')))
            cmd = (recon_data[info_data][options]['commands'].replace('${{ out_dir }}', 
                    out_file).replace('${{ target }}', target))
            command_info[descr] = cmd 
    
    return command_info

\end{minted}

\subsubsubsection{recon\_config.yaml}
\par La configuració veurem que hi ha unes variables \$\{\{ nom \}\} que em permeten fer el replace en el codi de python3. He trobat varies formes pero aquesta era la que més s'ajustaba el que volia. 
\begin{minted}
[
frame=lines,
framesep=2mm,
baselinestretch=1.2,
bgcolor=white,
fontsize=\footnotesize,
linenos
]
{markdown}
```yaml
full_scanner:
  nmap_quick_open:
    description: "Quick open ports with nmap"
    commands: | 
        nmap -F --open -oN ${{ out_dir }} ${{ target }} 
  
  nmap_quick_versions:
    description: "Quick scan with nmap"
    commands: | 
        nmap -sV -sC --open --version-all -oN ${{ out_dir }} ${{ target }} 
  
  nmap_full: 
    attack_technique: nmap
    description: "Scanning all ports with nmap and return service information"
    commands: | 
        nmap -sS -sV -sC -O -p- --min-rate 5000 -n -oN ${{ out_dir }} ${{ target }} 
  nmap_fast_udp:
    attack_technique: nmap
    description: "Scanning udp ports"
    commands: |  
        nmap -F -sU -sV -T 4 -oN ${{ out_dir }} ${{ target }} 
  
  masscan_full:
    attack_technique: masscan
    description: "Scanning all ports with masscan"
    commands: |  
        masscan --rate 10000 -p1-65535 --only-open -oL ${{ out_dir }} ${{ target }}
## Not ready yet
nmap_custom:
  nmap_allports:
    attack_technique: nmap
    description: "Scanning all ports with nmap"
    commands: | 
        nmap -p- --min-rate 5000 -n -oG ${{ out_dir }} ${{ target }} 
  
  nmap_ports:
    attack_technique: nmap
    description: "Scanning specific ports with nmap"
    commands: |  
        nmap -p ${{ ports }} -sC -sV ${{ out_dir }} ${{ target }}
```
\end{minted}
A continuació mostro el que fa la funcio \cite{3} argsparse, i ens retorna la seguent llista: 
\par
{'threads': 100, 'verbose': False, 'path': '/tmp/autorecon/', 'services': 'recon', 'target': ['127.0.0.1'], 'all_ports': True, 'ports': None}
\par
\subsubsubsection{web.py}\par
En l'apartat web, trobem que el funcionament es molt similar. L'unic que previament pots passar-l'hi el parametre ssl i et trasnforma el http a https. 

\begin{minted}
[
frame=lines,
framesep=2mm,
baselinestretch=1.2,
bgcolor=white,
fontsize=\footnotesize,
linenos
]
{python}
#!/usr/bin/env python3
### This script enum all ports, services and vulnerabilities with nmap
import yaml
import os 
## need port, user, passwd, type scanner, CMS, fuzzing, layer (http/https), wordlist, url 

def web(all_info, recon_path, out_path):
    target = all_info['target']
    info_data = all_info['scanner']
    port = all_info['port']
    url_path = all_info['file_path']
    cms = all_info['cms']


    # Check https or http scheme 
    scheme = 'http'
    if all_info['ssl']:
       scheme = 'https' 

    url = '{0}://{1}:{2}{3}'.format(scheme,target, port, url_path) 
    
    ## Set wordlist
    wordlist = all_info['wordlist']
    
    if wordlist == None:
        wordlist = '/usr/share/seclists/Discovery/Web-Content/big.txt'
    

    # get path information
    web_data = os.path.join(recon_path, "web_config.yaml")
    command_info = {} 
    


    with open(web_data, 'r') as unparsed:
        try:
            web_data = yaml.safe_load(unparsed)
        except yaml.YAMLError as exc:
            print(exc)
        
        list_data = web_data[info_data].keys()
        
        for options in list_data:
            descr = web_data[info_data][options]['description']
            out_file = os.path.join(out_path, '.'.join((options, 'txt')))
            cmd = (web_data[info_data][options]['commands'].replace('${{ out_dir }}', 
                    out_file).replace('${{ target }}', target).replace('${{ port }}', 
                    port).replace('${{ url }}', url).replace('${{ wordlist }}',wordlist))
            command_info[descr] = cmd 
    
    return command_info

\end{minted}


\subsubsubsection{web\_config.yaml}
\begin{minted}
[
frame=lines,
framesep=2mm,
baselinestretch=1.2,
bgcolor=white,
fontsize=\footnotesize,
linenos
]
{markdown}
```yaml
full_scanner:
  nmap_scan:
    attack_technique: nmap
    description: "Scanning nmap with scripts banner,(http* or ssl*) and not 
        (brute or broadcast or dos or external or http-slowloris* or fuzzer)"
    commands: |
      nmap -sV -p ${{ port }} --script="banner,(http*·or·ssl*)·and·not·
        (brute·or·broadcast·or·dos·or·external·or·http-slowloris*·or·fuzzer)" -oN 
            ${{ out_dir }} ${{ target }}
  
  curl_index: 
    attack_technique: curl-index
    description: "Enum index web content"
    commands: | 
        curl -sSik ${{ url }} -m 10 | tee ${{ out_dir }}
  
  whatweb_index: 
    attack_technique: whatweb
    description: "Enum versions webs"
    commands: | 
        whatweb -v -a 3 ${{ target }} 
  nikto_scanenr:
    attack_technique: nikto
    description: "Nikto detect misconfiguration, risky files, etc."
    commands: |  
        nikto -h ${{ url }}  | tee ${{ out_dir }}
  
  fuzzing_ffuf:
    attack_technique: ffuf
    description: "Fuzzing directories and files"
    commands: |  
        ffuf -w ${{ wordlist }} -u ${{ url }}FUZZ -c -e '.php,.asp,.aspx,.txt,.html,.zip' 
            -o ${{ out_dir }}
wp: 
    wordpres_scanner: 
        attack_technique: wpscan
        description: "Fuzzing directories and files"
        commands: |  
            wpscan --url ${{ url }} -e vp,vt,tt,cb,dbe,u,m --plugins-detection aggressive 
                --plugins-version-detection aggressive -o ${{ out_dir }}
```

\end{minted}



\subsubsubsection{smb.py}\par
Aquest script ja veurem que el que fa és diferent de la resta. Per arguments és passen uns tags que seran buscats en el fitxer de configuració. Fa referència a la informació que té l'atacant. Es a dir, si l'atacant no té informació, el més normal seria iniciar per una sessió nul·la. Una vegada ja es te credencials doncs és passa a enumerar aquesta informació amb l'usuari i contrasenya. I també hi ha l'opció de passar-li el hash, aquesta et permet de la mateixa forma executar comandes de forma remota com si tinguessis la contrasenya.

\begin{minted}
[
frame=lines,
framesep=2mm,
baselinestretch=1.2,
bgcolor=white,
fontsize=\footnotesize,
linenos
]
{python}
#!/usr/bin/env python3
import yaml
import os 

# list only data pased 
def get_tags(smb_data, tags):
    
    list_data = smb_data.keys()
    final_services = [] 

    for parameter in tags:
        for technique in list_data:
            total = 0
            for opt in smb_data[technique]['tags']: 
                ecual = False
                if parameter.lower() == opt.lower(): 
                    ecual = True
                    total +=1
                if not ecual:
                    break
            if total == len(tags):
                final_services.append(technique)

    return final_services

#return command

def smb(all_info, recon_path, out_path):
    target = all_info['target']
    port = all_info['port']
    tags = all_info['tags'].split(',')
    username = all_info['username']
    password = all_info['password']
    
    if len(username) == 0 or len(password) == 0: 
        username = ''
        password = ''

    # get path information
    smb_data = os.path.join(recon_path, "smb_config.yaml")
    command_info = {} 
     
    with open(smb_data, 'r') as unparsed:
        try:
            smb_data = yaml.safe_load(unparsed)
        except yaml.YAMLError as exc:
            print(exc)
        
        services = get_tags(smb_data, tags)        
        
        for options in services:
            descr = smb_data[options]['description']
            out_file = os.path.join(out_path, '.'.join((options, 'txt')))
            cmd = (smb_data[options]['commands'].replace('${{ out_dir }}', 
            out_file).replace('${{ target }}', target).replace('${{ port }}', 
            port).replace('${{ username }}', username).replace('${{ password }}', password))
            command_info[descr] = cmd 
         
    return command_info
\end{minted}

\subsubsubsection{smb\_config.yaml}
\begin{minted}
[
frame=lines,
framesep=2mm,
baselinestretch=1.2,
bgcolor=white,
fontsize=\footnotesize,
linenos
]
{markdown}
```yaml
smbc_null_session:
    attack_technique: smbclient
    description: "Samba null sessions enum directories"
    tags: 
        - nocreds
    commands: |
        smbclient -L \\\\${{ target }}\\ -N -p ${{ port }}
cme_null_session:
    attack_technique: crackmapexec
    description: "Enum null sessions with Crackmapexec"
    tags: 
        - nocreds
    commands: |
        crackmapexec smb ${{ target }} -u '' -p '' --shares --timeout 5
enum4linux_null_session:
    attack_technique: Enum4linux
    description: "Enum null sessions with enum4linux"
    tags: 
        - nocreds
    commands: |
        enum4linux -a ${{ target }} | tee ${{ out_dir }} 
cme_dsa_session:
    attack_technique: crackmapexec
    description: "Enum directories and see if have shell"
    tags: 
        - creds
    commands: |
        crackmapexec smb ${{ target }} -u '${{ username }}' -p '${{ password }}' 
            --shares --timeout 5
cme_passthehash:
    attack_technique: crackmapexec
    description: "Enum directories and see if have shell"
    tags: 
        - hash
    commands: |
        crackmapexec smb ${{ target }} -u '${{ username }}' -H '${{ hash }}' 
            --shares --timeout 5

```

\end{minted}

\subsubsubsection{ldap.py}

\begin{minted}
[
frame=lines,
framesep=2mm,
baselinestretch=1.2,
bgcolor=white,
fontsize=\footnotesize,
linenos
]
{python}
    #!/usr/bin/env python3
import yaml
import os 
import ldap3

# Not ready yet
def enum_ldap(server, port):
    server = ldap3.Server(server,port, use_ssl = True)
    connection = ldap3.Connection(server)
    if connection.bind(): 
        info = server.info
    else:
        info = False

def get_tags(ldap_data, tags):
    
    list_data = ldap_data.keys()
    final_services = [] 

    for parameter in tags:
        for technique in list_data:
            total = 0
            for opt in ldap_data[technique]['tags']: 
                ecual = False
                if parameter.lower() == opt.lower(): 
                    ecual = True
                    total +=1
                if not ecual:
                    break
            if total == len(tags):
                final_services.append(technique)

    return final_services

def ldap(all_info, recon_path, out_path):
    target = all_info['target']
    port = all_info['port']
    tags = all_info['tags'].split(',')
    username = all_info['username']
    password = all_info['password']
    
    if len(username) == 0 or len(password) == 0: 
        username = ''
        password = ''

    # get path information
    ldap_data = os.path.join(recon_path, "ldap_config.yaml")
    command_info = {} 
     
    with open(ldap_data, 'r') as unparsed:
        try:
            ldap_data = yaml.safe_load(unparsed)
        except yaml.YAMLError as exc:
            print(exc)
        
        services = get_tags(ldap_data, tags)        
        
        for options in services:
            descr = ldap_data[options]['description']
            out_file = os.path.join(out_path, '.'.join((options, 'txt')))
            cmd = (ldap_data[options]['commands'].replace('${{ out_dir }}', 
            out_file).replace('${{ target }}', target).replace('${{ port }}', 
            port).replace('${{ username }}', username).replace('${{ password }}', password))
            command_info[descr] = cmd 
         
    return command_info

\end{minted}

\subsubsubsection{ldap\_config.yaml}
\begin{minted}
[
frame=lines,
framesep=2mm,
baselinestretch=1.2,
bgcolor=white,
fontsize=\footnotesize,
linenos
]
{markdown}
```yaml
ldap_nmap_session:
    attack_technique: nmap
    description: "Enum null sessions with nmap"
    tags: 
        - nocreds
    commands: |
        nmap -sV -p ${{ port }} --script="banner,(ldap* or ssl*) and not 
        (brute or broadcast or dos or external or fuzzer)" -oN "${{ out_dir }}" ${{ target }}
```

\end{minted}

\subsubsubsection{ftp}
\begin{minted}
[
frame=lines,
framesep=2mm,
baselinestretch=1.2,
bgcolor=white,
fontsize=\footnotesize,
linenos
]
{python}
    #!/usr/bin/env python3
import yaml
import os 

def get_tags(ftp_data, tags):

    list_data = ftp_data.keys()
    final_services = [] 

    for parameter in tags:
        for technique in list_data:
            total = 0
            for opt in ftp_data[technique]['tags']: 
                ecual = False
                if parameter.lower() == opt.lower(): 
                    ecual = True
                    total +=1
                if not ecual:
                    break
            if total == len(tags):
                final_services.append(technique)

    return final_services

def ftp(all_info, recon_path, out_path):
    target = all_info['target']
    port = all_info['port']
    tags = all_info['tags'].split(',')
    username = all_info['username']
    password = all_info['password']
    
    if len(username) == 0 or len(password) == 0: 
        username = 'anonymous'
        password = ''

    # Check ftp or ftps scheme 
    scheme = 'ftp'
    if all_info['ssl']:
       scheme = 'ftps' 
    
    # get path information
    ftp_data = os.path.join(recon_path, "ftp_config.yaml")
    command_info = {} 
     
    with open(ftp_data, 'r') as unparsed:
        try:
            ftp_data = yaml.safe_load(unparsed)
        except yaml.YAMLError as exc:
            print(exc)
        
        services = get_tags(ftp_data, tags)        
        
        for options in services:
            descr = ftp_data[options]['description']
            out_file = os.path.join(out_path, '.'.join((options, 'txt')))
            cmd = (ftp_data[options]['commands'].replace('${{ out_dir }}', 
            out_file).replace('${{ target }}', target).replace('${{ port }}', 
            port).replace('${{ username }}', username).replace('${{ password }}', 
            password).replace('${{ scheme }}', scheme))
            command_info[descr] = cmd 
         
    return command_info

\end{minted}
\subsubsubsection{ftp\_config.yaml}
\begin{minted}
[
frame=lines,
framesep=2mm,
baselinestretch=1.2,
bgcolor=white,
fontsize=\footnotesize,
linenos
]
{markdown}
```yaml
ftp_anon_session:
    attack_technique: nmap_ftp
    description: "check anonymous user ftp"
    tags: 
        - nocreds
    commands: |
        nmap -sC -sV -p ${{ port }} -on ${{ out_dir }} ${{ target }}
donwload_anonymous_session:
    attack_technique: download_files
    description: "downloading all files"
    tags: 
        - nocreds
    commands: |
        wget -r ${{ scheme }}://${{ target }}:${{ port }} -P ${{ out_dir }}
donwload_creds_session:
    attack_technique: download_files
    description: "downloading all files"
    tags: 
        - creds
    commands: |
        wget -r ${{ scheme }}://${{ username }}:${{ password }}@${{ target 
        }}:${{ port }} -P ${{ out_dir }}
```

\end{minted}


\subsection{Utilització}

Una vegada mostrada, la part més important del codi, procediré en explicar l'ús de l'eina. 
Primer de tot mostraré les opcions que té l'eina. 


\begin{minted}
[
baselinestretch=1.2,
bgcolor=white,
fontsize=\footnotesize,
]
{text}
usage: AutoGPenT0ols.py [-h] [-t THREADS] [--verbose] [--show-commands] [--path PATH] 
{recon,web,smtp,smb,ftp,ldap,snmp} ...

                           #                                                 
                           ##                                ###             
  ###### #######  ######## ###  ## ######## ####### #######  ###         ####
 ###           ##          #### ##    ###   ##   ##       ## ###        ###  
 ###  ##  ######   ####### #######    ###   ##   ##  ##   ## ###        ###  
 ###  ##  ###      ###     ### ###    ###   ##   ##  ##   ## ###        ###  
  ######  ###      ####### ###  ##    ###   #######   #####  ####### #####   
                                 #                                           

                                 Version: 0.1
                                 Codename: CapitanJ4ck

optional arguments:
  -h, --help            show this help message and exit
  -t THREADS            set how many concurrent threads to use (default: 100)
  --verbose             enable verbose output
  --show-commands       Just check commands 
  --path PATH           Destination path (default: /tmp/gpt_report)

services:
  available options

  {recon,web,smtp,smb,ftp,ldap,snmp}
    recon               Initial recon
    web                 Web server scanner
    smtp                smtp enumeration
    smb                 Enum smb
    ftp                 FTP enum
    ldap                LDAP enum
    snmp                Enum SNMP

We are in... Let the hacking begin!

\end{minted}

La metodologia que hauriem de seguir es realiar en primer lloc un recon, aquesta funció ens permet enumerar a nivell de xarxa tots els ports que poden esta oberts. Utilitza tant nmap i masscan. Son eines molt potents que et permeten enumerar la xarxa de forma ràpida i precisa. 


\begin{minted}
[
baselinestretch=1.2,
bgcolor=white,
fontsize=\footnotesize,
]
{text}

usage: AutoGPenT0ols.py recon [-h] [--scanner [SCANNER]] 
[--all-ports] [--ports [PORTS]] [--full] [target]

positional arguments:
  target               (Target Required *) The target IP(s), range(s), 
  CIDR(s), hostname(s), FQDN(s), file(s) containing a list of targets

optional arguments:
  -h, --help           show this help message and exit
  --scanner [SCANNER]  Select scanner (default : full_scanner)
  --all-ports          scan all ports
  --ports [PORTS]      scan specific ports
  --full               Full recon scan

\end{minted}

Una vegada tenim la primera informació podem porcedir en enumerar els serveis oberts. Jo inicio l'explicació en funcio del port, de menys a més gran. 
Iniciem pel port 22 - FTP. En aquest cas veurem que ens permet realitzar les següents operacions. 

\begin{minted}
[
baselinestretch=1.2,
bgcolor=white,
fontsize=\footnotesize,
]
{text}
gpt ftp -h
usage: AutoGPenT0ols.py ftp [-h] [-u [USERNAME]] [-p [PASSWORD]] [-H [HASH]] 
[--tags [TAGS]] [--ssl] [--port [PORT]] [target]

positional arguments:
  target         (Target Required *) The target IP(s), range(s),
                  CIDR(s), hostname(s), FQDN(s), file(s) containing a list of targets

optional arguments:
  -h, --help     show this help message and exit
  -u [USERNAME]  username(s) or file(s) containing usernames
  -p [PASSWORD]  password(s) or file(s) containing passwords
  -H [HASH]      Pass The hash
  --tags [TAGS]  What do you have? [Creds, NoCreds, Hash, Shell] (Default: NoCreds)
  --ssl          usage of SSL/TLS requests
  --port [PORT]  scan specific port (Default 22)

\end{minted}

Seguim amb el port 80/443 - Web \par
El que ens permet es escollir el tipo de scanner en aquest cas per defecte hi ha el full\_scanner pero podriem escollir el wp que es per una web wordpress. També veiem que ens permet passar la les credencials i la wordlist. Per defecte ja estan configurades aquestes variables per evitar errors. I el parametre ssl et permet afegir el https en la url. 
\begin{minted}
[
baselinestretch=1.2,
bgcolor=white,
fontsize=\footnotesize,
]
{text}

gpt web -h

usage: AutoGPenT0ols.py web [-h] [-u [USERNAME]] [-p [PASSWORD]] [-H [HASH]] [-w WORDLIST [WORDLIST ...]] 
[--scanner [SCANNER]] [--ssl] [--port [PORT]] [--file-path [FILE_PATH]] [--cms [CMS]] [target]

positional arguments:
  target                (Target Required *) The target IP(s), range(s), 
                        CIDR(s), hostname(s), FQDN(s), file(s) containing a list of targets

optional arguments:
  -h, --help            show this help message and exit
  -u [USERNAME]         username(s) or file(s) containing usernames
  -p [PASSWORD]         password(s) or file(s) containing passwords
  -H [HASH]             Pass The hash
  -w WORDLIST [WORDLIST ...]
                        set wordlist (Default SecList wordlist)
  --scanner [SCANNER]   Select scanner (default : full_scanner)
  --ssl                 usage of SSL/TLS requests
  --port [PORT]         scan specific port (Default 80)
  --file-path [FILE_PATH]
                        Specify to find the requested resource and start the enumeration 
                        with that route (Default / )
  --cms [CMS]           What do you have? [Wordpress, Joombla, Drupal] (Default: None)

\end{minted}
\par
El servei LDAP, ens permet obtenir molta informació de la victima. Aquest protocol es molt utilizat per Controladors de Domini. 

\begin{minted}
[
baselinestretch=1.2,
bgcolor=white,
fontsize=\footnotesize,
]
{text}

gpt ldap -h

usage: AutoGPenT0ols.py ldap [-h] [-u [USERNAME]] [-p [PASSWORD]] 
    [-H [HASH]] [--tags [TAGS]] [--port [PORT]] [target]

positional arguments:
  target         (Target Required *) The target IP(s), range(s), 
                CIDR(s), hostname(s), FQDN(s), file(s) containing a list of targets

optional arguments:
  -h, --help     show this help message and exit
  -u [USERNAME]  username(s) or file(s) containing usernames
  -p [PASSWORD]  password(s) or file(s) containing passwords
  -H [HASH]      Pass The hash
  --tags [TAGS]  What do you have? [Creds, NoCreds, Hash, Shell] (Default: NoCreds)
  --port [PORT]  scan specific port (Default 389)

\end{minted}

Finalment trobem el protocol samba, aquest es el que s'utilitza per les carpetas compartides. El funcionament es similar els anteriors. Pases un tag, en cas que no tinguis informació previa ho deixes per defecte i extraurà tota la informació possible d'aquest protocol. 

\begin{minted}
[
baselinestretch=1.2,
bgcolor=white,
fontsize=\footnotesize,
]
{text}

gpt smb -h

usage: AutoGPenT0ols.py smb [-h] [-u [USERNAME]] [-p [PASSWORD]] 
        [-H [HASH]] [--tags [TAGS]] [--port [PORT]] [target]

positional arguments:
  target         (Target Required *) The target IP(s), range(s), 
                  CIDR(s), hostname(s), FQDN(s), file(s) containing a list of targets

optional arguments:
  -h, --help     show this help message and exit
  -u [USERNAME]  username(s) or file(s) containing usernames
  -p [PASSWORD]  password(s) or file(s) containing passwords
  -H [HASH]      Pass The hash
  --tags [TAGS]  What do you have? [Creds, NoCreds, Hash, Shell] (Default: NoCreds)
  --port [PORT]  scan specific port (Default 445)

\end{minted}

\subsection{Exemples d'ús}
Com he especificat anteriorment primer haurem d'executar l'opció de recon. Un exemple seria el següent: 

\begin{minted}
[
baselinestretch=1.2,
bgcolor=white,
fontsize=\footnotesize,
]
{text}

╭─[internetghost2] as jack in ~                                                                                                                                                                                      
╰──➤  gpt --verbose recon 192.168.1.12 
[+] Managment: Directory /tmp/gpt_report/192.168.1.12/recon already exist
[↘] recon: nmap -sV -sC --open --version-all -oN 
    /tmp/gpt_report/192.168.1.12/recon/nmap_quick_versions.txt 192.168.1.12 
-- 𝒞𝒽𝑒𝒸𝓀 𝒢𝒫𝑒𝓃𝒯𝒪𝑜𝓁𝓈 --
[+] Quick open ports with nmap: Succesfully!
Starting Nmap 7.91 ( https://nmap.org ) at 2021-05-30 08:28 UTC
Nmap scan report for internetghost2.home (192.168.1.12)
Host is up (0.000021s latency).
Not shown: 95 closed ports
PORT    STATE SERVICE
21/tcp  open  ftp
80/tcp  open  http
139/tcp open  netbios-ssn
443/tcp open  https
445/tcp open  microsoft-ds

Nmap done: 1 IP address (1 host up) scanned in 0.32 seconds

[O] Quick scan with nmap: Running
Starting Nmap 7.91 ( https://nmap.org ) at 2021-05-30 08:28 UTC
Nmap scan report for internetghost2.home (192.168.1.12)
Host is up (0.000012s latency).
Not shown: 993 closed ports
PORT      STATE SERVICE     VERSION
21/tcp    open  ftp         Pure-FTPd
80/tcp    open  http
| fingerprint-strings: 
|   FourOhFourRequest: 
|     HTTP/1.0 302 Found
|     Location: https://http:443:///nice%20ports%2C/Tri%6Eity.txt%2ebak
|     Date: Sun, 30 May 2021 08:28:58 GMT
|     Content-Length: 5
|     Content-Type: text/plain; charset=utf-8
|     Found
|   GenericLines, Hello, Help, Kerberos, RTSPRequest, SSLSessionReq, 
    SSLv23SessionReq, TLSSessionReq, TerminalServerCookie: 
|     HTTP/1.1 400 Bad Request
|     Content-Type: text/plain; charset=utf-8
|     Connection: close
|     Request
|   GetRequest, HTTPOptions: 
|     HTTP/1.0 302 Found
|     Location: https://http:443:///
|     Date: Sun, 30 May 2021 08:28:53 GMT
|     Content-Length: 5
|     Content-Type: text/plain; charset=utf-8
|_    Found
|_http-title: Did not follow redirect to https://internetghost2.home:443/
139/tcp   open  netbios-ssn Samba smbd 4.6.2
443/tcp   open  ssl/http    Golang net/http server (Go-IPFS json-rpc or InfluxDB API)
|_http-title: Site doesn't have a title (text/plain; charset=utf-8).
| ssl-cert: Subject: commonName=TRAEFIK DEFAULT CERT
| Subject Alternative Name: 
DNS:5fb207ee8c5e95258740a5c37d0c43c8.de0ab1663e5dab751b09fa03961c958a.traefik.default
| Not valid before: 2021-05-30T08:07:59
|_Not valid after:  2022-05-30T08:07:59
|_ssl-date: TLS randomness does not represent time
| tls-alpn: 
|   h2
|   http/1.1
|_  acme-tls/1
| tls-nextprotoneg: 
|   h2
|   http/1.1
|_  acme-tls/1
445/tcp   open  netbios-ssn Samba smbd 4.6.2
8090/tcp  open  http        PHP cli server 5.5 or later (PHP 7.4.9)
|_http-title: Login - Adminer
|_http-trane-info: Problem with XML parsing of /evox/about
30000/tcp open  tcpwrapped
1 service unrecognized despite returning data. If you know the service/version, 
please submit the following fingerprint at https://nmap.org/cgi-bin/submit.cgi?new-service :
SF-Port80-TCP:V=7.91%I=9%D=5/30%Time=60B34CC5%P=x86_64-pc-linux-gnu%r(GetR
SF:equest,9C,"HTTP/1\.0\x20302\x20Found\r\nLocation:\x20https://http:443:/
SF://\r\nDate:\x20Sun,\x2030\x20May\x202021\x2008:28:53\x20GMT\r\nContent-
SF:Length:\x205\r\nContent-Type:\x20text/plain;\x20charset=utf-8\r\n\r\nFo
SF:und")%r(HTTPOptions,9C,"HTTP/1\.0\x20302\x20Found\r\nLocation:\x20https
SF:://http:443:///\r\nDate:\x20Sun,\x2030\x20May\x202021\x2008:28:53\x20GM
SF:T\r\nContent-Length:\x205\r\nContent-Type:\x20text/plain;\x20charset=ut
SF:f-8\r\n\r\nFound")%r(RTSPRequest,67,"HTTP/1\.1\x20400\x20Bad\x20Request
SF:\r\nContent-Type:\x20text/plain;\x20charset=utf-8\r\nConnection:\x20clo
SF:se\r\n\r\n400\x20Bad\x20Request")%r(FourOhFourRequest,BF,"HTTP/1\.0\x20
SF:302\x20Found\r\nLocation:\x20https://http:443:///nice%20ports%2C/Tri%6E
SF:ity\.txt%2ebak\r\nDate:\x20Sun,\x2030\x20May\x202021\x2008:28:58\x20GMT
SF:\r\nContent-Length:\x205\r\nContent-Type:\x20text/plain;\x20charset=utf
SF:-8\r\n\r\nFound")%r(GenericLines,67,"HTTP/1\.1\x20400\x20Bad\x20Request
SF:\r\nContent-Type:\x20text/plain;\x20charset=utf-8\r\nConnection:\x20clo
SF:se\r\n\r\n400\x20Bad\x20Request")%r(Hello,67,"HTTP/1\.1\x20400\x20Bad\x
SF:20Request\r\nContent-Type:\x20text/plain;\x20charset=utf-8\r\nConnectio
SF:n:\x20close\r\n\r\n400\x20Bad\x20Request")%r(Help,67,"HTTP/1\.1\x20400\
SF:x20Bad\x20Request\r\nContent-Type:\x20text/plain;\x20charset=utf-8\r\nC
SF:onnection:\x20close\r\n\r\n400\x20Bad\x20Request")%r(SSLSessionReq,67,"
SF:HTTP/1\.1\x20400\x20Bad\x20Request\r\nContent-Type:\x20text/plain;\x20c
SF:harset=utf-8\r\nConnection:\x20close\r\n\r\n400\x20Bad\x20Request")%r(T
SF:erminalServerCookie,67,"HTTP/1\.1\x20400\x20Bad\x20Request\r\nContent-T
SF:ype:\x20text/plain;\x20charset=utf-8\r\nConnection:\x20close\r\n\r\n400
SF:\x20Bad\x20Request")%r(TLSSessionReq,67,"HTTP/1\.1\x20400\x20Bad\x20Req
SF:uest\r\nContent-Type:\x20text/plain;\x20charset=utf-8\r\nConnection:\x2
SF:0close\r\n\r\n400\x20Bad\x20Request")%r(SSLv23SessionReq,67,"HTTP/1\.1\
SF:x20400\x20Bad\x20Request\r\nContent-Type:\x20text/plain;\x20charset=utf
SF:-8\r\nConnection:\x20close\r\n\r\n400\x20Bad\x20Request")%r(Kerberos,67
SF:,"HTTP/1\.1\x20400\x20Bad\x20Request\r\nContent-Type:\x20text/plain;\x2
SF:0charset=utf-8\r\nConnection:\x20close\r\n\r\n400\x20Bad\x20Request");

Host script results:
| smb2-security-mode: 
|   2.10: 
|_    Message signing enabled but not required
|_smb2-time: Protocol negotiation failed (SMB2)

Service detection performed. Please report any incorrect results at https://nmap.org/submit/ .
Nmap done: 1 IP address (1 host up) scanned in 266.04 seconds

[.] Scanning all ports with nmap and return service information: Running
Starting Nmap 7.91 ( https://nmap.org ) at 2021-05-30 08:33 UTC
Nmap scan report for 192.168.1.12
Host is up (0.00013s latency).
Not shown: 65517 closed ports
PORT      STATE SERVICE     VERSION
21/tcp    open  ftp         Pure-FTPd
80/tcp    open  http
| fingerprint-strings: 
|   FourOhFourRequest: 
|     HTTP/1.0 302 Found
|     Location: https://http:443:///nice%20ports%2C/Tri%6Eity.txt%2ebak
|     Date: Sun, 30 May 2021 08:33:25 GMT
|     Content-Length: 5
|     Content-Type: text/plain; charset=utf-8
|     Found
|   GenericLines, Help, Kerberos, LDAPSearchReq, LPDString, RTSPRequest, 
SSLSessionReq, TLSSessionReq, TerminalServerCookie: 
|     HTTP/1.1 400 Bad Request
|     Content-Type: text/plain; charset=utf-8
|     Connection: close
|     Request
|   GetRequest, HTTPOptions: 
|     HTTP/1.0 302 Found
|     Location: https://http:443:///
|     Date: Sun, 30 May 2021 08:33:20 GMT
|     Content-Length: 5
|     Content-Type: text/plain; charset=utf-8
|_    Found
|_http-title: Did not follow redirect to https://192.168.1.12:443/
139/tcp   open  netbios-ssn Samba smbd 4.6.2
443/tcp   open  ssl/http    Golang net/http server (Go-IPFS json-rpc or InfluxDB API)
|_http-title: Site doesn't have a title (text/plain; charset=utf-8).
| ssl-cert: Subject: commonName=TRAEFIK DEFAULT CERT
| Subject Alternative Name: 
DNS:5fb207ee8c5e95258740a5c37d0c43c8.de0ab1663e5dab751b09fa03961c958a.traefik.default
| Not valid before: 2021-05-30T08:07:59
|_Not valid after:  2022-05-30T08:07:59
|_ssl-date: TLS randomness does not represent time
| tls-alpn: 
|   h2
|   http/1.1
|_  acme-tls/1
| tls-nextprotoneg: 
|   h2
|   http/1.1
|_  acme-tls/1
445/tcp   open  netbios-ssn Samba smbd 4.6.2
8090/tcp  open  http        PHP cli server 5.5 or later (PHP 7.4.9)
|_http-title: Login - Adminer
|_http-trane-info: Problem with XML parsing of /evox/about
30000/tcp open  tcpwrapped
30001/tcp open  tcpwrapped
30002/tcp open  tcpwrapped
30003/tcp open  tcpwrapped
30004/tcp open  tcpwrapped
30005/tcp open  tcpwrapped
30006/tcp open  tcpwrapped
30007/tcp open  tcpwrapped
30008/tcp open  tcpwrapped
30009/tcp open  tcpwrapped
43677/tcp open  unknown
| fingerprint-strings: 
|   NULL, RPCCheck: 
|_    {"type":"Tier1","version":"1.0"}
57621/tcp open  unknown
2 services unrecognized despite returning data. If you know the service/version, 
please submit the following fingerprints at https://nmap.org/cgi-bin/submit.cgi?new-service :
==============NEXT SERVICE FINGERPRINT (SUBMIT INDIVIDUALLY)==============
SF-Port80-TCP:V=7.91%I=7%D=5/30%Time=60B34DD0%P=x86_64-pc-linux-gnu%r(GetR
SF:equest,9C,"HTTP/1\.0\x20302\x20Found\r\nLocation:\x20https://http:443:/
SF://\r\nDate:\x20Sun,\x2030\x20May\x202021\x2008:33:20\x20GMT\r\nContent-
SF:Length:\x205\r\nContent-Type:\x20text/plain;\x20charset=utf-8\r\n\r\nFo
SF:und")%r(HTTPOptions,9C,"HTTP/1\.0\x20302\x20Found\r\nLocation:\x20https
SF:://http:443:///\r\nDate:\x20Sun,\x2030\x20May\x202021\x2008:33:20\x20GM
SF:T\r\nContent-Length:\x205\r\nContent-Type:\x20text/plain;\x20charset=ut
SF:f-8\r\n\r\nFound")%r(RTSPRequest,67,"HTTP/1\.1\x20400\x20Bad\x20Request
SF:\r\nContent-Type:\x20text/plain;\x20charset=utf-8\r\nConnection:\x20clo
SF:se\r\n\r\n400\x20Bad\x20Request")%r(FourOhFourRequest,BF,"HTTP/1\.0\x20
SF:302\x20Found\r\nLocation:\x20https://http:443:///nice%20ports%2C/Tri%6E
SF:ity\.txt%2ebak\r\nDate:\x20Sun,\x2030\x20May\x202021\x2008:33:25\x20GMT
SF:\r\nContent-Length:\x205\r\nContent-Type:\x20text/plain;\x20charset=utf
SF:-8\r\n\r\nFound")%r(GenericLines,67,"HTTP/1\.1\x20400\x20Bad\x20Request
SF:\r\nContent-Type:\x20text/plain;\x20charset=utf-8\r\nConnection:\x20clo
SF:se\r\n\r\n400\x20Bad\x20Request")%r(Help,67,"HTTP/1\.1\x20400\x20Bad\x2
SF:0Request\r\nContent-Type:\x20text/plain;\x20charset=utf-8\r\nConnection
SF::\x20close\r\n\r\n400\x20Bad\x20Request")%r(SSLSessionReq,67,"HTTP/1\.1
SF:\x20400\x20Bad\x20Request\r\nContent-Type:\x20text/plain;\x20charset=ut
SF:f-8\r\nConnection:\x20close\r\n\r\n400\x20Bad\x20Request")%r(TerminalSe
SF:rverCookie,67,"HTTP/1\.1\x20400\x20Bad\x20Request\r\nContent-Type:\x20t
SF:ext/plain;\x20charset=utf-8\r\nConnection:\x20close\r\n\r\n400\x20Bad\x
SF:20Request")%r(TLSSessionReq,67,"HTTP/1\.1\x20400\x20Bad\x20Request\r\nC
SF:ontent-Type:\x20text/plain;\x20charset=utf-8\r\nConnection:\x20close\r\
SF:n\r\n400\x20Bad\x20Request")%r(Kerberos,67,"HTTP/1\.1\x20400\x20Bad\x20
SF:Request\r\nContent-Type:\x20text/plain;\x20charset=utf-8\r\nConnection:
SF:\x20close\r\n\r\n400\x20Bad\x20Request")%r(LPDString,67,"HTTP/1\.1\x204
SF:00\x20Bad\x20Request\r\nContent-Type:\x20text/plain;\x20charset=utf-8\r
SF:\nConnection:\x20close\r\n\r\n400\x20Bad\x20Request")%r(LDAPSearchReq,6
SF:7,"HTTP/1\.1\x20400\x20Bad\x20Request\r\nContent-Type:\x20text/plain;\x
SF:20charset=utf-8\r\nConnection:\x20close\r\n\r\n400\x20Bad\x20Request");
==============NEXT SERVICE FINGERPRINT (SUBMIT INDIVIDUALLY)==============
SF-Port43677-TCP:V=7.91%I=7%D=5/30%Time=60B34DCA%P=x86_64-pc-linux-gnu%r(N
SF:ULL,22,"{\"type\":\"Tier1\",\"version\":\"1\.0\"}\r\n")%r(RPCCheck,22,"
SF:{\"type\":\"Tier1\",\"version\":\"1\.0\"}\r\n");
Device type: general purpose
Running: Linux 4.X|5.X
OS CPE: cpe:/o:linux:linux_kernel:4 cpe:/o:linux:linux_kernel:5
OS details: Linux 4.15 - 5.6
Network Distance: 1 hop

Host script results:
| smb2-security-mode: 
|   2.10: 
|_    Message signing enabled but not required
|_smb2-time: Protocol negotiation failed (SMB2)

OS and Service detection performed. Please report any incorrect results at https://nmap.org/submit/ .
Nmap done: 1 IP address (1 host up) scanned in 99.70 seconds

[+] Scanning udp ports: Succesfully!
Starting Nmap 7.91 ( https://nmap.org ) at 2021-05-30 08:34 UTC
Nmap scan report for internetghost2.home (192.168.1.12)
Host is up (0.00014s latency).
Not shown: 81 open|filtered ports
PORT      STATE  SERVICE        VERSION
17/udp    closed qotd
69/udp    closed tftp
88/udp    closed kerberos-sec
139/udp   closed netbios-ssn
500/udp   closed isakmp
515/udp   closed printer
593/udp   closed http-rpc-epmap
1025/udp  closed blackjack
1029/udp  closed solid-mux
1433/udp  closed ms-sql-s
1813/udp  closed radacct
5000/udp  closed upnp
31337/udp closed BackOrifice
32769/udp closed filenet-rpc
32815/udp closed unknown
49154/udp closed unknown
49186/udp closed unknown
49193/udp closed unknown
49200/udp closed unknown

Service detection performed. Please report any incorrect results at https://nmap.org/submit/ .
Nmap done: 1 IP address (1 host up) scanned in 277.81 seconds

\end{minted}



\begin{minted}
[
baselinestretch=1.2,
bgcolor=white,
fontsize=\footnotesize,
]
{text}
╰──➤  gpt --verbose smb 192.168.1.12 -h
usage: AutoGPenT0ols.py smb [-h] [-u [USERNAME]] [-p [PASSWORD]] [-H [HASH]] [--tags [TAGS]]
                            [--port [PORT]]
                            [target]

positional arguments:
  target         (Target Required *) The target IP(s), range(s), CIDR(s), hostname(s), FQDN(s),
                 file(s) containing a list of targets

optional arguments:
  -h, --help     show this help message and exit
  -u [USERNAME]  username(s) or file(s) containing usernames
  -p [PASSWORD]  password(s) or file(s) containing passwords
  -H [HASH]      Pass The hash
  --tags [TAGS]  What do you have? [Creds, NoCreds, Hash, Shell] (Default: NoCreds)
  --port [PORT]  scan specific port (Default 445)
[+] Managment: Directory /tmp/gpt_report/192.168.1.12/smb already exist
[/] smb: enum4linux -a 192.168.1.12 | 
tee /tmp/gpt_report/192.168.1.12/smb/enum4linux_null_session.txt 
[+] Managment: Directory /tmp/gpt_report/192.168.1.12/smb already exist
[/] smb: enum4linux -a 192.168.1.12 | 
tee /tmp/gpt_report/192.168.1.12/smb/enum4linux_null_session.txt 
-- 𝒞𝒽𝑒𝒸𝓀 𝒢𝒫𝑒𝓃𝒯𝒪𝑜𝓁𝓈 --
[+] Samba null sessions enum directories: Succesfully!

    Sharename       Type      Comment
    ---------       ----      -------
    marc            Disk      Carpeta de l'Alumne marc
    michael         Disk      Carpeta de l'Alumne michael
    ernest          Disk      Carpeta de l'Alumne ernest
    cosmin          Disk      Carpeta de l'Alumne cosmin
    isaac           Disk      Carpeta de l'Alumne isaac
    jordi           Disk      Carpeta de l'Alumne jordi
    fernando        Disk      Carpeta de l'Alumne fernando
    victor          Disk      Carpeta de l'Alumne victor
    lina            Disk      Carpeta de l'Alumne lina
[\] jack            Disk      Carpeta de l'Alumne jack
    ju              Disk      Carpeta de l'Alumne ju
    andre           Disk      Carpeta de l'Alumne andre
    anton           Disk      Carpeta de l'Alumne anton
    juan            Disk      Carpeta de l'Alumne juan
    pe              Disk      Carpeta de l'Alumne pe
    IPC$            IPC       IPC Service (Samba Server)
SMB1 disabled -- no workgroup available

[-] Enum null sessions with Crackmapexec: Something wrong
[*] First time use detected
[*] Creating home directory structure
[*] Creating default workspace
[*] Initializing SSH protocol database
[*] Initializing WINRM protocol database
[*] Initializing MSSQL protocol database
[*] Initializing SMB protocol database
[*] Initializing LDAP protocol database
[*] Copying default configuration file
[*] Generating SSL certificate

[|] Enum null sessions with enum4linux: Running

SMB1 disabled -- no workgroup available

[-] Enum null sessions with Crackmapexec: Something wrong
[*] First time use detected
[*] Creating home directory structure
[*] Creating default workspace
[*] Initializing SSH protocol database
[*] Initializing WINRM protocol database
[*] Initializing MSSQL protocol database
[*] Initializing SMB protocol database
[*] Initializing LDAP protocol database
[*] Copying default configuration file
[-] Generating SSL certificate
Starting enum4linux v0.8.9 ( http://labs.portcullis.co.uk/application/enum4linux/ ) 
on Sun May 30 08:28:05 2021

 ========================== 
|    Target Information    |
 ========================== 
Target ........... 192.168.1.12
RID Range ........ 500-550,1000-1050
Username ......... ''
Password ......... ''
Known Usernames .. administrator, guest, krbtgt, domain admins, root, bin, none


 ==================================================== 
|    Enumerating Workgroup/Domain on 192.168.1.12    |
 ==================================================== 
[E] Can't find workgroup/domain


 ============================================ 
|    Nbtstat Information for 192.168.1.12    |
 ============================================ 
Looking up status of 192.168.1.12
No reply from 192.168.1.12

 ===================================== 
|    Session Check on 192.168.1.12    |
 ===================================== 
[+] Server 192.168.1.12 allows sessions using username '', password ''
[+] Got domain/workgroup name: 

 =========================================== 
|    Getting domain SID for 192.168.1.12    |
 =========================================== 
Domain Name: MYGROUP
Domain Sid: (NULL SID)
[+] Can't determine if host is part of domain or part of a workgroup

 ====================================== 
|    OS information on 192.168.1.12    |
 ====================================== 
[+] Got OS info for 192.168.1.12 from smbclient: 
[+] Got OS info for 192.168.1.12 from srvinfo:
    1C144B1916EA   Wk Sv PrQ Unx NT SNT Samba Server
    platform_id     :    500
    os version      :    6.1
    server type     :    0x809a03

 ============================= 
|    Users on 192.168.1.12    |
 ============================= 
index: 0x1 RID: 0x3f5 acb: 0x00000010 Account: juan    Name: Linux User    Desc: 
index: 0x2 RID: 0x3f6 acb: 0x00000010 Account: pe    Name: Linux User    Desc: 
index: 0x3 RID: 0x3e8 acb: 0x00000010 Account: marc    Name: Linux User    Desc: 
index: 0x4 RID: 0x3ea acb: 0x00000010 Account: ernest    Name: Linux User    Desc: 
index: 0x5 RID: 0x3ec acb: 0x00000010 Account: isaac    Name: Linux User    Desc: 
index: 0x6 RID: 0x3e9 acb: 0x00000010 Account: michael    Name: Linux User    Desc: 
index: 0x7 RID: 0x3eb acb: 0x00000010 Account: cosmin    Name: Linux User    Desc: 
index: 0x8 RID: 0x3ed acb: 0x00000010 Account: jordi    Name: Linux User    Desc: 
index: 0x9 RID: 0x3ee acb: 0x00000010 Account: fernando    Name: Linux User    Desc: 
index: 0xa RID: 0x3ef acb: 0x00000010 Account: victor    Name: Linux User    Desc: 
index: 0xb RID: 0x3f0 acb: 0x00000010 Account: lina    Name: Linux User    Desc: 
index: 0xc RID: 0x3f1 acb: 0x00000010 Account: jack    Name: Linux User    Desc: 
index: 0xd RID: 0x3f2 acb: 0x00000010 Account: ju    Name: Linux User    Desc: 
index: 0xe RID: 0x3f3 acb: 0x00000010 Account: andre    Name: Linux User    Desc: 
index: 0xf RID: 0x3f4 acb: 0x00000010 Account: anton    Name: Linux User    Desc: 

user:[juan] rid:[0x3f5]
user:[pe] rid:[0x3f6]
user:[marc] rid:[0x3e8]
user:[ernest] rid:[0x3ea]
user:[isaac] rid:[0x3ec]
user:[michael] rid:[0x3e9]
user:[cosmin] rid:[0x3eb]
user:[jordi] rid:[0x3ed]
user:[fernando] rid:[0x3ee]
user:[victor] rid:[0x3ef]
user:[lina] rid:[0x3f0]
user:[jack] rid:[0x3f1]
user:[ju] rid:[0x3f2]
user:[andre] rid:[0x3f3]
user:[anton] rid:[0x3f4]

 ========================================= 
|    Share Enumeration on 192.168.1.12    |
 ========================================= 

    Sharename       Type      Comment
    ---------       ----      -------
    marc            Disk      Carpeta de l'Alumne marc
    michael         Disk      Carpeta de l'Alumne michael
    ernest          Disk      Carpeta de l'Alumne ernest
    cosmin          Disk      Carpeta de l'Alumne cosmin
    isaac           Disk      Carpeta de l'Alumne isaac
    jordi           Disk      Carpeta de l'Alumne jordi
    fernando        Disk      Carpeta de l'Alumne fernando
    victor          Disk      Carpeta de l'Alumne victor
    lina            Disk      Carpeta de l'Alumne lina
    jack            Disk      Carpeta de l'Alumne jack
    ju              Disk      Carpeta de l'Alumne ju
    andre           Disk      Carpeta de l'Alumne andre
    anton           Disk      Carpeta de l'Alumne anton
    juan            Disk      Carpeta de l'Alumne juan
    pe              Disk      Carpeta de l'Alumne pe
    IPC$            IPC       IPC Service (Samba Server)
SMB1 disabled -- no workgroup available

[+] Attempting to map shares on 192.168.1.12
//192.168.1.12/marc    Mapping: DENIED, Listing: N/A
//192.168.1.12/michael    Mapping: DENIED, Listing: N/A
//192.168.1.12/ernest    Mapping: DENIED, Listing: N/A
//192.168.1.12/cosmin    Mapping: DENIED, Listing: N/A
//192.168.1.12/isaac    Mapping: DENIED, Listing: N/A
//192.168.1.12/jordi    Mapping: DENIED, Listing: N/A
//192.168.1.12/fernando    Mapping: DENIED, Listing: N/A
//192.168.1.12/victor    Mapping: DENIED, Listing: N/A
//192.168.1.12/lina    Mapping: DENIED, Listing: N/A
//192.168.1.12/jack    Mapping: DENIED, Listing: N/A
//192.168.1.12/ju    Mapping: DENIED, Listing: N/A
//192.168.1.12/andre    Mapping: DENIED, Listing: N/A
//192.168.1.12/anton    Mapping: DENIED, Listing: N/A
//192.168.1.12/juan    Mapping: DENIED, Listing: N/A
//192.168.1.12/pe    Mapping: DENIED, Listing: N/A
//192.168.1.12/IPC$    [E] Can't understand response:
NT_STATUS_OBJECT_NAME_NOT_FOUND listing \*

 ==================================================== 
|    Password Policy Information for 192.168.1.12    |
 ==================================================== 


[+] Attaching to 192.168.1.12 using a NULL share

[+] Trying protocol 139/SMB...

    [!] Protocol failed: ('unpack requires a buffer of 1 bytes', 
    "When unpacking field 'SecurityMode | <B | b''[:1]'")

[+] Trying protocol 445/SMB...

[+] Found domain(s):

    [+] 1C144B1916EA
    [+] Builtin

[+] Password Info for Domain: 1C144B1916EA

    [+] Minimum password length: 5
    [+] Password history length: None
    [+] Maximum password age: 37 days 6 hours 21 minutes 
    [+] Password Complexity Flags: 000000

        [+] Domain Refuse Password Change: 0
        [+] Domain Password Store Cleartext: 0
        [+] Domain Password Lockout Admins: 0
        [+] Domain Password No Clear Change: 0
        [+] Domain Password No Anon Change: 0
        [+] Domain Password Complex: 0

    [+] Minimum password age: None
    [+] Reset Account Lockout Counter: 30 minutes 
    [+] Locked Account Duration: 30 minutes 
    [+] Account Lockout Threshold: None
    [+] Forced Log off Time: 37 days 6 hours 21 minutes 


[+] Retieved partial password policy with rpcclient:

Password Complexity: Disabled
Minimum Password Length: 5


 ============================== 
|    Groups on 192.168.1.12    |
 ============================== 

[+] Getting builtin groups:

[+] Getting builtin group memberships:

[+] Getting local groups:

[+] Getting local group memberships:

[+] Getting domain groups:

[+] Getting domain group memberships:

 ======================================================================= 
|    Users on 192.168.1.12 via RID cycling (RIDS: 500-550,1000-1050)    |
 ======================================================================= 
[I] Found new SID: S-1-22-1
[I] Found new SID: S-1-5-21-2112836514-3653563689-2825713174
[I] Found new SID: S-1-5-32
[+] Enumerating users using SID S-1-5-32 and logon username '', password ''
S-1-5-32-544 BUILTIN\Administrators (Local Group)
S-1-5-32-545 BUILTIN\Users (Local Group)
S-1-5-32-546 BUILTIN\Guests (Local Group)
S-1-5-32-547 BUILTIN\Power Users (Local Group)
S-1-5-32-548 BUILTIN\Account Operators (Local Group)
S-1-5-32-549 BUILTIN\Server Operators (Local Group)
S-1-5-32-550 BUILTIN\Print Operators (Local Group)
[+] Enumerating users using SID S-1-5-21-2112836514-3653563689-2825713174 
and logon username '', password ''
S-1-5-21-2112836514-3653563689-2825713174-500 *unknown*\*unknown* (8)
S-1-5-21-2112836514-3653563689-2825713174-501 1C144B1916EA\nobody (Local User)
S-1-5-21-2112836514-3653563689-2825713174-513 1C144B1916EA\None (Domain Group)
S-1-5-21-2112836514-3653563689-2825713174-1000 1C144B1916EA\marc (Local User)
S-1-5-21-2112836514-3653563689-2825713174-1001 1C144B1916EA\michael (Local User)
S-1-5-21-2112836514-3653563689-2825713174-1002 1C144B1916EA\ernest (Local User)
S-1-5-21-2112836514-3653563689-2825713174-1003 1C144B1916EA\cosmin (Local User)
S-1-5-21-2112836514-3653563689-2825713174-1004 1C144B1916EA\isaac (Local User)
S-1-5-21-2112836514-3653563689-2825713174-1005 1C144B1916EA\jordi (Local User)
S-1-5-21-2112836514-3653563689-2825713174-1006 1C144B1916EA\fernando (Local User)
S-1-5-21-2112836514-3653563689-2825713174-1007 1C144B1916EA\victor (Local User)
S-1-5-21-2112836514-3653563689-2825713174-1008 1C144B1916EA\lina (Local User)
S-1-5-21-2112836514-3653563689-2825713174-1009 1C144B1916EA\jack (Local User)
S-1-5-21-2112836514-3653563689-2825713174-1010 1C144B1916EA\ju (Local User)
S-1-5-21-2112836514-3653563689-2825713174-1011 1C144B1916EA\andre (Local User)
S-1-5-21-2112836514-3653563689-2825713174-1012 1C144B1916EA\anton (Local User)
S-1-5-21-2112836514-3653563689-2825713174-1013 1C144B1916EA\juan (Local User)
S-1-5-21-2112836514-3653563689-2825713174-1014 1C144B1916EA\pe (Local User)
[+] Enumerating users using SID S-1-22-1 and logon username '', password ''
S-1-22-1-1000 Unix User\marc (Local User)
S-1-22-1-1001 Unix User\michael (Local User)
S-1-22-1-1002 Unix User\ernest (Local User)
S-1-22-1-1003 Unix User\cosmin (Local User)
S-1-22-1-1004 Unix User\isaac (Local User)
S-1-22-1-1005 Unix User\jordi (Local User)
S-1-22-1-1006 Unix User\fernando (Local User)
S-1-22-1-1007 Unix User\victor (Local User)
S-1-22-1-1008 Unix User\lina (Local User)
S-1-22-1-1009 Unix User\jack (Local User)
S-1-22-1-1010 Unix User\ju (Local User)
S-1-22-1-1011 Unix User\andre (Local User)
S-1-22-1-1012 Unix User\anton (Local User)
S-1-22-1-1013 Unix User\juan (Local User)
S-1-22-1-1014 Unix User\pe (Local User)

 ============================================= 
|    Getting printer info for 192.168.1.12    |
 ============================================= 
Could not initialise spoolss. Error was NT_STATUS_OBJECT_NAME_NOT_FOUND


enum4linux complete on Sun May 30 08:28:39 2021


\end{minted}

A nivell de web ve a mostra una mica el mateix, l'ús de l'eina no es molt complexa i extreu molta informació encara que sigui utilitzant enumeració bàsica. 

\begin{minted}
[
baselinestretch=1.2,
bgcolor=white,
fontsize=\footnotesize,
]
{text}

[-] Enum index web content: Something wrong
HTTP/1.1 302 Found
Location: https://192.168.1.12:443/
Date: Sun, 30 May 2021 09:31:26 GMT
Content-Length: 5
Content-Type: text/plain; charset=utf-8

Found
[b] Enum versions webs: Running
WhatWeb report for http://192.168.1.12
Status    : 302 Found
Title     : <None>
IP        : 192.168.1.12
Country   : RESERVED, ZZ

Summary   : RedirectLocation[https://192.168.1.12:443/]

Detected Plugins:
[ RedirectLocation ]
    HTTP Server string location. used with http-status 301 and 
    302 

    String       : https://192.168.1.12:443/ (from location)

HTTP Headers:
    HTTP/1.1 302 Found
    Location: https://192.168.1.12:443/
    Date: Sun, 30 May 2021 09:31:27 GMT
    Content-Length: 5
    Content-Type: text/plain; charset=utf-8
    Connection: close
    
WhatWeb report for https://192.168.1.12/
Status    : 404 Not Found
Title     : <None>
IP        : 192.168.1.12
Country   : RESERVED, ZZ

Summary   : UncommonHeaders[x-content-type-options]

Detected Plugins:
[ UncommonHeaders ]
    Uncommon HTTP server headers. The blacklist includes all 
    the standard headers and many non standard but common ones. 
    Interesting but fairly common headers should have their own 
    plugins, eg. x-powered-by, server and x-aspnet-version. 
    Info about headers can be found at www.http-stats.com 

    String       : x-content-type-options (from headers)

HTTP Headers:
    HTTP/1.1 404 Not Found
    Content-Type: text/plain; charset=utf-8
    X-Content-Type-Options: nosniff
    Date: Sun, 30 May 2021 09:31:28 GMT
    Content-Length: 19
    Connection: close
    

[+] Nikto detect misconfiguration, risky files, etc.: Succesfully!
- Nikto v2.1.6
---------------------------------------------------------------------------
+ Target IP:          192.168.1.12
+ Target Hostname:    192.168.1.12
+ Target Port:        80
+ Start Time:         2021-05-30 09:31:30 (GMT0)
---------------------------------------------------------------------------
+ Server: No banner retrieved
+ The anti-clickjacking X-Frame-Options header is not present.
+ The X-XSS-Protection header is not defined. This header can hint to the user agent to protect against some forms of XSS
+ The X-Content-Type-Options header is not set. This could allow the user agent to render the content of the site in a different fashion to the MIME type
+ Root page / redirects to: https://192.168.1.12:443/
+ No CGI Directories found (use '-C all' to force check all possible dirs)
+ 7915 requests: 0 error(s) and 3 item(s) reported on remote host
+ End Time:           2021-05-30 09:31:42 (GMT0) (12 seconds)
---------------------------------------------------------------------------
+ 1 host(s) tested

[-] Fuzzing directories and files: Command timeout
[-] SIGINT or CTRL-C detected.: Exiting gracefully

\end{minted}
\begin{minted}
[
baselinestretch=1.2,
bgcolor=white,
fontsize=\footnotesize,
]
{text}


\end{minted}
    \section{Conclusions}

El principal objectiu d'aquest treball era tenir una eina que em permetés automatitzar l'enumeració d'un pentest, també ha set aprendre i entendre com desenvolupar una eina. Considerant que havia posat el llisto molt alt però no comptava amb el poc temps que he tingut. De totes maneres em sento molt content d'haver escollit aquest projecte, ja que m'ha ajudat molt a millorar aspectes tècnics de la programació i entendre el funcionament d'altres eines.
\par
Per això crec que és un bon inici per seguir desenvolupant aquestes habilitats, considero que crear-te les teves eines per automatitzar tasques et pot aportar molta més eficàcia i t'obliga a entendre molt més a baix nivell la tasca que s'ha de realitzar.
\par
Se que encara li falta molt perquè l'eina deixi de ser una versió beta però realment he aconseguit donar-li forma i a poc a poc podré anar implementant les millores que m'he proposat.
\subsection{Línies de futur}
Com he dit, jo vull seguir millorant les habilitats adquirides en aquest projecte, per tant, seguiré treballant per teure'l de la fase beta i deixar-lo d'utilitzar en entorns de laboratori. 

\subsection{Problemes trobats}

Per una banda, ha set bastant complicat el tema d'utilitzar alguns mòduls perquè no havia utilitzat mai, com ara el argsparse i subproces. Aquest dos son els que més m'ha costat implementar. Finalment he aconseguit fer que funcionin però en algunes comandes no ho agafa del tot bé i no executa la comanda.

Un altre problema que vaig trobar que no hi contava era el tema de modificar els fitxers de configuració pensava que seria més fàcil i únicament hauria de buscar una llibreria que m'ho permetis però finalment ho vaig haver de preguntar quines alternatives podria utilitzar.
\pagebreak

\begin{thebibliography}{9}
\bibitem{cme} 
CrackMapExec is developed by @byt3bl33d3r and maintained by @mpgn
\textit{Is a post-exploitation tool that helps automate assessing the security of large Active Directory networks}
https://mpgn.gitbook.io/crackmapexec/

\bibitem{autorecon} 
Tib3rius 
\textit{Penetration tester with 10 years experience, specializing in web application security. I have an MSc in #InfoSec, created AutoRecon and two popular PrivEsc courses for the #OSCP labs/exam. He's stream on Twitch and create #CyberSecurity}
https://github.com/Tib3rius/AutoRecon


\bibitem{argparse} 
argparse
\texttt{Parser for command-line options, arguments and sub-commands}
https://docs.python.org/3/library/argparse.html

\bibitem{subproces} 
subprocess
\texttt{Subprocess management}
https://docs.python.org/3/library/subprocess.html

\bibitem{os}
os.path
\texttt{Common pathname manipulations}
https://docs.python.org/3/library/os.path.html

\end{thebibliography}

\end{document}


